\documentclass{IEEEtran}
\usepackage[ngerman]{babel}
\usepackage[utf8]{inputenc}
\usepackage[T1]{fontenc}
\usepackage{graphicx}
\usepackage{natbib}
\usepackage{url}
\usepackage{hyperref}
\usepackage{todonotes}

%allow more space between the words to prevent overfull boxes
\setlength{\emergencystretch}{1em}

\bibliographystyle{abbrvnat}
\setcitestyle{authoryear,round}

\title{RSTensorFlow: Analyse und Optimierung\\(vorläufiger Titel)}
\author{Michael Engel}
\date{\today}

\begin{document}
\maketitle

\section{Abstract}
\label{sec:abstract}
\begin{itemize}
	\item{Zusammenfassender Abstract}
\end{itemize}
\section{Einleitung}
\label{sec:einleitung}
In den vergangenen Jahren haben sich mobile Geräte, wie beispielsweise Smartphones, zu Assistenten des Alltags entwickelt. Auf diesen kleinen Helfern werden viele komplexe Anwendungen und Funktionen genutzt, welche verschiedenste Deep Learning Modelle verwenden. So werden beispielsweise Deep Recurrent Networks Neural Networks zur Spracherkennung \cite{bspSpracherkennung} genutzt. Zur Erkennung von Objekten werden Deep Convolutional Networks eingesetzt \cite{bspObjekterkennung}. Deep Learning Modelle führen viele sehr rechenintensive Berechnungen durch. Deshalb wird bevorzugt eine Art Client-Server-Anwendung umgesetzt, welche das Deep Learning Modell auf einem separaten Server ausführt. Die mobilen Clients schicken über eine entsprechende App die benötigten Eingangsdaten für das Netzwerk an den Server und dieser schickt das Ergebnis des Deep Learning Netzwerks zurück an den Client. Dies ist sowohl zeit- als auch kostenintensiv. Um Deep Learning Netzwerke direkt auf dem Mobilgerät zu verwenden, bieten Frameworks wie TensorFlow (\url{https://www.tensorflow.org/}) eine eigene Variante der Software an. Jedoch nutzen diese nicht alle Rechenressourcen des Endgeräts, sondern lediglich die CPU. Für größere Applikationen ist dies oft nicht ausreichend, um in angemessener Zeit ein adäquates Ergebnis zu erhalten. Projekte wie RSTensorFlow \cite{rstensorflow2017} versuchen daher eine Unterstützung der GPU in das bestehende OpenSource-Framework TensorFlow zu integrieren und den Erfolg anhand geeigneter Metriken zu bewerten. \\
Die Veröffentlichung zum Projekt \textit{RSTensorFlow} \cite{rstensorflow2017} dient als zentrale Arbeit dieses Papers und stellt den Ausgangspunkt dar. Die in \cite{rstensorflow2017} beschriebenen Experimente werden in diesem Paper nachgestellt und für andere Geräte, welche nicht direkt von Google hergestellt wurden, durchgeführt. Über die angepasste \textit{TF Classify} App werden die Ausführungszeiten gemessen. Ebenso wird mit Hilfe der \textit{Trepn Profiler} App die Auslastung der CPU und GPU überprüft. Diese Anwendung dient ebenfalls dem Aufzeichnen weiterer Metriken, wie beispielsweise der Auslastung des Speichers. Diese wurde bereits in \cite{rstensorflow2017} als ein mögliches Bottleneck hinsichtlich der Performance vermutet. Für eine Optimierung der Conv2D-Operation wird das sogenannte Unfolding näher betrachtet. In \cite{conv2d-optimizing-unfolding} sowie in \cite{Rajbhandari:2017:OCM:3037697.3037745} werden mit dieser Technik Convolutional Networks optimiert. Hierfür wägen beide Arbeiten das Potential des Unfoldings ab und erläutern die Funktionsweise. 
\\ 

\section{Related Work}
\label{sec:relatedwork}
In \cite{rstensorflow2017} wird die derzeit noch eher mangelhafte Unterstützung für tiefe neuronale Netze auf mobilen Endgeräten erläutert. Die meisten Deep Learning Frameworks bieten zwar eine Variante für Mobilgeräte, können jedoch lediglich die CPU als Recheneinheit nutzen. Aktuelle Forschungen beschäftigen sich damit, diese Frameworks für Mobilgeräte zu erweitern, sodass weitere Komponenten für die komplexen Berechnungen tiefer neuronaler Netze genutzt werden können. Hier steht insbesondere die GPU als weitere Rechenressource im Fokus der Forschung für unterschiedlichste Anwendungen neuronaler Netze. Um die GPU nutzen zu können, wird von verschiedenen Forschungsprojekten entweder OpenCL \cite{opencl} oder RenderScript \cite{renderscript} verwendet. Im Bereich der  Continuous Vision Applikationen wird in \cite{deepmon2017} ein GPU-basiertes Deep Learning Framework vorgestellt. Hier wird zur Nutzung der GPU OpenCL \cite{opencl} in Kombination mit Vulkan \cite{vulkan} eingesetzt. Ebenso greift wird in \cite{deepsense2016} auf OpenCL \cite{opencl} zurückgegriffen, um ein Framework für Deep Convolutional Neural Networks für mobile Geräte mit GPU-Unterstützung umzusetzen. Ein weiteres Forschungsprojekt ist CNNDroid \cite{cnndroid2016}. Bei der Implementierung CNNDroid handelt es sich um eine OpenSource-Bibliothek für trainierte CNN auf Android-Geräten. Im Gegensatz zu den anderen Projekten integriert RSTensorFlow die GPU-Unterstützung in das beliebte, bereits bestehende Deep Learning Framework \textit{TensorFlow}. \\
Für die Evaluierung des umgesetzten Frameworks werden von allen vorgestellten Forschungsprojekten Metriken wie die Ausführungszeit betrachtet. Diese Arbeit greift die allgemeine Vorgehensweise der in \cite{rstensorflow2017} vorgestellten Experimente auf und wendet dies für andere Geräte an. Damit soll eine Validierung der in \cite{rstensorflow2017} vorgestellten Ergebnisse erzielt werden. Die nicht performante Nutzung des Arbeitsspeichers von RenderScript wird von den Autoren von \cite{rstensorflow2017} als ein möglicher Grund für die verschlechterte Ausführungszeit bei der Conv2D-Operation vermutet. Daher werden in diesem Paper weitere Metriken, wie beispielsweise die Speicherauslastung, betrachtet. 
\section{RSTensorFlow}
\label{sec:rstensorflow}
+ was ist rstensorflow?
	+ tensorflow, modifiziert, damit renderscript für rechenintensive operationen verwendet wird
+ wie wird es installiert?
+ was wird installiert? 3 Apps, kurz erklären... relevant nur die classify app, überleitung zu den operationen (wobei anpassung am kernel == alle apps von betroffen?)

\subsection{Grobe Architektur}
\label{subsec:architektur}
+ welche files (.cc, .h, .java etc) sind relevant (also wo wurden anpassungen für rstensorflow umgesetzt?)
+ diagramme

%ggf dieses kapitel streichen	
\subsection{Modifizierte Operationen}
\label{subsec:modoperationen}
+ Welche Operationen wurden damit implementiert? Jeweils kurz erklären, was diese Operation macht
	+ matmul 
	+ conv2d
		- \url{http://cs231n.github.io/convolutional-networks/}



\section{Vorbereitungen}
\label{sec:vorbereitungen}
In \cite{rstensorflow2017} wurden die durchgeführten Experimente wurden eigene Tests 


Für die Rekonstruktion wird zunächst analyse inkl. build prozess dann rek. experiment

Bei TensorFlow handelt es sich um ein umfangreiches und komplexes Projekt, weshalb eigene Anpassungen ein gutes Verständnis dieses Deep Learning Framworks erfordern. 

Bei TensorFlow handelt es sich um ein umfangreiches und komplexes Projekt, weshalb eigene Anpassungen ein gutes Verständnis dieses Deep Learning Framworks erfordern. Daher wird zunächst eine Analyse des Quellcodes im Hinblick auf die von RSTensorFlow vorgenommenen Anpassungen in \ref{subsec:quellcodeanalyse} untersucht. 
\\

+ versucht die prebuilds von rstensorflow (welche auf der homepage/github bereit gestellt sind) zu verwenden
	- für bestmögliche vergleichbarkeit bzw rekonstruktion der experimente
	- lediglich die eigen-prebuild (ohne rs) ist nutzbar
	- die rs-prebuild schmiert ab
+ nach längerer suche, konnte festgestellt werden, dass eine .so nicht geladen werden konnte
+ es handelt sich dabei um die .so, welche im rahmen des rstf projekts umgesetzt wurde
+ daher ist das eigenständige kompilieren von rstensorflow bzw. der android demo nötig was in \ref{subsec:buildprozess} dargestellt wird
+ optional: dies wird gleichzeitig als vorbereitung für die optimierung der conv2d betrachtet



\subsection{Analyse des Quellcodes}
\label{subsec:quellcodeanalyse}
+ welche files (.cc, .h, .java etc) sind relevant (also wo wurden anpassungen für rstensorflow umgesetzt?)
+ diagramme		
+ App-Struktur erklären 
	1 App wird in Java geschrieben und ververwendet einen NDKHelper, welcher jni nutzt
	2 In der libtensorflow\_inference.so ist:
		1 NDK in jni enthalten gibt aufruf weiter an 2
		2 Tensorflow in c++ gibt aufruf weiter an 3 
		3 RSWrapper in c++ gibt aufruf weiter an 4
		4 Renderscript welche wieder auf zugreift auf 3
	3 RSRuntime in .so 
+ das rstf projekt bietet auf der homepage die libtensorflowInference.so an - mit und ohne rs
	- für anpassungen muss allerdings selbst die libtensorflowInference.so kompiliert werden
+ RSRuntime so's durch extra projekt kompilierbar
	- librs.mScriptConv.so usw 
	

+ umfangreiche log-Ausgaben fehlen -> daher muss quellcode angepasst und gebaut werden, das bauen in \ref{subsec:buildprozess}

\subsection{Der Build-Prozess}
\label{subsec:buildprozess}
+ tools - bazel, android sdk, ndk
+ Bild mit build-struktur (siehe präsentation von 07.12.)
+ RSRuntime so's werden nur eingebunden, nicht kompiliert
	- hier fehlt die librs.mScriptConv.so -> musste selbst kompiliert und eingebunden werden
+ nachdem technische aspekte geklärt sind, wird versucht die experimente aus \cite{rstensorflow2017} in \ref{subsec:rekonstruktion} zu rekonstruieren 

\section{Experimente}
\label{sec:experimente}
+ Kurze Einleitung, was in dem Kapitel gemacht wird
+ Hardware-Tabelle (inkl. die Hardware aus \cite{rstensorflow2017})

\subsection{Verwendete Metriken}
\label{subsec:metriken}
Mögliche weitere Metriken wie in Kapitel 7 von \cite{deepmon2017} beschrieben ggf. verwenden
\begin{itemize}
	\item{Ist Metrik für dieses Projekt sinnvoll?}
	\item{Ist Metrik für dieses Projekt umsetzbar (Aufwand etc.)?}
\end{itemize}

Verwendete Metrik in \cite{rstensorflow2017}: Ausführungszeit
	+ im Code von tensorflow integriert 
	+ gibt auskunft über die Zeit einer einzelnen matmul-/conv2d operation und eines kompletten pfades
	+ über adb logcat -s TF\_ANDROID\_LOG können diese in eine Datei gelenkt werden

Metrik CPU Load
	+ über trepn profiler
	+ gibt auskunft über die auslastung der ressource
	
Metrik GPU Load
	+ über trepn profiler
	+ gibt auskunft über die auslastung der ressource

Metrik RAM (Memory Space):
	+ über trepn profiler
	+ nicht straight forward, da von Trepn Profiler RAM des kompletten Systems gemessen wird
	+ daher System im Ruhezustand profilen (inkl. laufender Trepn app) und (durchschnittlichen) RAM ermitteln und vom später gemessenen abziehen

\subsection{Aufbau des Experiments}
\label{subsec:aufbauexperiment}
+ Software
	+ Mit was werden die Daten erzeugt?
		- trepn profiler (app) -> Einstellungen definieren und speichern!
		- log ausgaben (direkt von tensorflow demo) -> siehe \ref{subsec:anpassungentf}
+ Hardware
	+ ggf. abbildung?
Beschreibung des Experiment-Aufbaus
\begin{itemize}
	\item{Wahl der Android-Geräte (Samsung J5 und ggf. weitere), deren Android-Version und Hardware)}
	\item{Wie werden die gewählten Metriken gemessen?}
\end{itemize}

\subsection{Anpassungen von TensorFlow}
\label{subsec:anpassungentf}
+ Die Trepn Profiler "out-of-the-box" aber die logausgaben müssen aufbereitet werden, da die vorhandenen logausgaben unzureichend sind
+ adb logging
	da nur "ein Stream" verfügbar - alles mit adb logcat TF\_ANDROID\_LOG in eine CSV-Datei
	Format: <Feld1>|<Feld n>|...
	Dabei ist Feld1 der Index, um was für eine Operation es sich handelt (da von classify sowohl conv2d als auch matmul ausgeführt wird)
	Für matmul:
		+ die matrizengröße
	Für conv2d:
		+ anzahl filter (== outdepth)
		+ stride		]
		+ padding		- Berechnung der Anzahl der matmul operationen -> hat einfluss auf ausführungszeit? ggf. mit unfolding doch verbesserbar? (für Ausblick dann)
		+ filtergröße	]
		+ inputgröße	]
		

\subsection{Durchführung}
\label{subsec:experimentdurchfuehrung}
+ trepn profiler: profile system
	- profiling erst im standard modus um die normale menge an memory zu erhalten
	- dann app starten

\subsection{Auswertung}
\label{subsec:experimentauswertung}
+ was kamen für ergebnisse heraus?
+ vergleich zu \cite{rstensorflow2017}
	- deutet auf eine generelle verschlechterung mit rs hin
	- direkte vergleichbarkeit aber leider nicht gegeben, da die genaue erhebung der daten unbekannt


\section{Optional: Optimierung der Conv2D-Operation}
\label{sec:conv2d}
\begin{itemize}
	\item{Kurze Beschreibung der Convolution-Operation (wie in \cite{stanford-CS231n})}
	\item{Ist-Analyse des Quellcodes der Conv2D-Operation}
	\item{Unfolding (\cite{conv2d-optimizing-unfolding}) als mögliche Optimierung von Conv2D für RSTensorFlow}
	%\begin{figure}[!t]
	%	\centering
	%	\includegraphics[width=3.0in]{simple-corr-model.png}
	%	\caption{A very simple model to describe the correlation between agile and safety. }
	%	\label{fig:simple-corr-model}
	%\end{figure}
	\item{Implementierung der (möglichen) Optimierung}
\end{itemize}

\section{Fazit und Ausblick}
\label{sec:ausblick}
Einige Probleme:
	+ erstmal zum laufen bekommn (von rstensorflow bereitgestellter mod. build funzt nicht, build prozess etc.)
	+ rekonstruktion der experimente
	Hier kann aus dem referenzpaper bzw. dessen website zitiert oder sich darauf bezogen werden: ist schweres zeug, rs nativ in tensorflow zu integrieren

Schlussfolgerung:
	+ nur weil es die gpu ist bzw. eine zusätzliche ressource, muss es nicht unbedingt schneller sein
	+ ggf derzeit einfach nicht ausgereift -> standards!
	+ oder renderscript einfach ungeeignet
\begin{itemize}
	\item{Fazit
		\begin{itemize}
			\item{Erzielt RSTensorFlow ähnliche Ergebnisse bei den hier getesteten Geräten im Vergleich zu den Nexus-Geräten, welche in \cite{rstensorflow2017} verwendet wurden?}
			\item{Konnte eine Verbesserung von Conv2D erzielt werden?}
		\end{itemize}
		}
	\item{Ausblick (oder Future Work)
		\begin{itemize}
			\item{TensorFlow Lite}
			\item{OpenCL statt RenderScript}
		\end{itemize}
		}
\end{itemize}



\section{Zukünftige Arbeiten}
\label{sec:futurework}
+ future work: unfolding umzusetzen (wie auswertung zeigt: sehr viele matmuls bei conv und damit ist nur noch 1 nötig -> performance-memory-tradeoff? eher unwahrscheinlich, da für conv2d und rs eh die tensoren kopiert werden und bei auswertung kaum veränderungen bei dem genutzten speicher zu sehen waren) und somit 


\bibliography{mybib}{}
\end{document}