\section{Abstract}
\label{sec:abstract}
Die Bedeutung mobiler Geräte ist in den vergangenen Jahren enorm gestiegen. So unterstützen diese kleinen Helfer den Anwender bei einer Vielzahl von Aufgaben im Alltag. Für viele dieser komplexen Anwendungen und Funktionen werden verschiedenste Deep Learning Modelle eingesetzt. Zur Spracherkennung sind beispielsweise Deep Recurrent Neural Networks die bevorzugte Wahl bei der Umsetzung. Solche Modelle führen allerdings sehr viele rechenintensive Berechnungen durch, weshalb diese bevorzugt als Client-Server-Anwendung umgesetzt werden. Das mobile Gerät schickt lediglich die Eingangsdaten an einen Server auf dem das Deep Learning Model ausgeführt wird. Zwar unterstützen einige Frameworks das Ausführen von Deep Learning Modellen auf mobilen Endgeräten, jedoch wird von diesen nur die CPU genutzt. 
Diese Arbeit untersucht RSTensorFlow, eine Modifikation des beliebten OpenSource-Frameworks TensorFlow, und versucht die hierfür durchgeführten Experimente zur Performance-Bewertung für andere mobile Geräte nachzustellen. Zunächst wird in diesem Paper sowohl RSTensorFlow und die in dem Projekt modifizierten Operationen als auch die von TensorFlow zur Verfügung gestellten Android-Apps näher vorgestellt. Bei eine technische Analyse von RSTensorFlow wird dann der Quellcode und der Build-Prozess von TensorFlow untersucht. Dabei werden auch die vorgenommen Änderungen von RSTensorFlow vorgestellt. Bevor die durchgeführten Experimente und dessen Resultate näher beschrieben werden, ist es nötig die von den Autoren durchgeführten Experimente zu rekonstruieren. Diese Rekonstruktion führt zur Definition von weiteren Metriken und Anpassungen am Quellcode von RSTensorFlow für die Experimente. 