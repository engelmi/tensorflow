\section{Fazit und Ausblick}
\label{sec:ausblick}
Einige Probleme:
	+ erstmal zum laufen bekommn (von rstensorflow bereitgestellter mod. build funzt nicht, build prozess etc.)
	+ rekonstruktion der experimente
	Hier kann aus dem referenzpaper bzw. dessen website zitiert oder sich darauf bezogen werden: ist schweres zeug, rs nativ in tensorflow zu integrieren

Schlussfolgerung:
	+ nur weil es die gpu ist bzw. eine zusätzliche ressource, muss es nicht unbedingt schneller sein
	+ ggf derzeit einfach nicht ausgereift -> standards!
	+ oder renderscript einfach ungeeignet
\begin{itemize}
	\item{Fazit
		\begin{itemize}
			\item{Erzielt RSTensorFlow ähnliche Ergebnisse bei den hier getesteten Geräten im Vergleich zu den Nexus-Geräten, welche in \cite{rstensorflow2017} verwendet wurden?}
			\item{Konnte eine Verbesserung von Conv2D erzielt werden?}
		\end{itemize}
		}
	\item{Ausblick (oder Future Work)
		\begin{itemize}
			\item{TensorFlow Lite}
			\item{OpenCL statt RenderScript}
		\end{itemize}
		}
\end{itemize}



\section{Zukünftige Arbeiten}
\label{sec:futurework}
+ future work: unfolding umzusetzen (wie auswertung zeigt: sehr viele matmuls bei conv und damit ist nur noch 1 nötig -> performance-memory-tradeoff? eher unwahrscheinlich, da für conv2d und rs eh die tensoren kopiert werden und bei auswertung kaum veränderungen bei dem genutzten speicher zu sehen waren) und somit 
