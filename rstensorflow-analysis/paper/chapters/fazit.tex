\section{Fazit und Ausblick}
\label{sec:ausblick}
In diesem Paper wurde RSTensorFlow, eine angepasste Variante der Machine Learning Bibliothek TensorFlow, vorgestellt und näher untersucht. Ebenso wurden die in \cite{rstensorflow2017} durchgeführten Experimente für die Matrixmultiplikation und die Convolution-Operation nachgestellt und weitere Metriken für die eigenen Experimente definiert. Hierfür wurden Quellcode-Anpassungen im Kernel von RSTensorFlow umgesetzt. 
\\
Die von den durchgeführten Experimenten erhobenen Daten widersprechen teilweise den in \cite{rstensorflow2017} präsentierten Ergebnissen. Während im durchgeführten Experiment für RenderScript bei der Matrixmultiplikation eine schlechtere Performance gemessen wurde, wurde in \cite{rstensorflow2017} eine Steigerung der Performance angegeben. Bei Convolution-Operation steigen die Ausführungszeiten auf dem Samsung J5, ähnlich zu denen Werten in \cite{rstensorflow2017}, bei erhöhen der Filterzahl. Das Samsung S7 hingegen erzielt mit RenderScript eine fast doppelt so schnelle Bearbeitung der Convolution-Operation. Der Grund für diese Performance-Steigerung kann nicht an der Nutzung der GPU durch RenderScript liegen, da in \cite{rstensorflow2017} vom Nexus 5X die GPU von RenderScript genutzt wurde und dennoch eine Verschlechterung der Performance zu messen war. Auch unter Verwendung der anderen Metriken, wie beispielsweise die CPU-Last, lassen sich hierzu keine sicheren Aussagen treffen.  
\\
Die erzielten Ergebnisse decken sich somit nicht komplett mit den Resultaten aus \cite{rstensorflow2017}. Dabei kann der exakte Grund für die Unterschiede nicht festgestellt werden, sodass lediglich Vermutungen geäußert werden können. Diese Arbeit hat jedoch gezeigt, dass sich RenderScript auf verschiedenen mobilen Geräten zu sehr unterschiedlich auswirkt. Daher ist RenderScript für den breiten Einsatz noch nicht geeignet, hat aber auch durchaus Potential für den Einsatz im Bereich Deep Learning auf mobilen Endgeräten. 