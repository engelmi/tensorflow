\section{Related Work}
\label{sec:relatedwork}
In \cite{rstensorflow2017} wird die derzeit noch eher mangelhafte Unterstützung für tiefe neuronale Netze auf mobilen Endgeräten erläutert. Die meisten Deep Learning Frameworks bieten zwar eine Variante für Mobilgeräte, können jedoch lediglich die CPU als Recheneinheit nutzen. Aktuelle Forschungen beschäftigen sich damit, diese Frameworks für Mobilgeräte zu erweitern, sodass weitere Komponenten für die komplexen Berechnungen tiefer neuronaler Netze genutzt werden können. Hier steht insbesondere die GPU als weitere Rechenressource im Fokus der Forschung für unterschiedlichste Anwendungen neuronaler Netze. Um die GPU nutzen zu können, wird von verschiedenen Forschungsprojekten entweder OpenCL \cite{opencl} oder RenderScript \cite{renderscript} verwendet. Im Bereich der  Continuous Vision Applikationen wird in \cite{deepmon2017} ein GPU-basiertes Deep Learning Framework vorgestellt. Hier wird zur Nutzung der GPU OpenCL \cite{opencl} in Kombination mit Vulkan \cite{vulkan} eingesetzt. Ebenso greift wird in \cite{deepsense2016} auf OpenCL \cite{opencl} zurückgegriffen, um ein Framework für Deep Convolutional Neural Networks für mobile Geräte mit GPU-Unterstützung umzusetzen. Ein weiteres Forschungsprojekt ist CNNDroid \cite{cnndroid2016}. Bei der Implementierung CNNDroid handelt es sich um eine OpenSource-Bibliothek für trainierte CNN auf Android-Geräten. Im Gegensatz zu den anderen Projekten integriert RSTensorFlow die GPU-Unterstützung in das beliebte, bereits bestehende Deep Learning Framework \textit{TensorFlow}. \\
Für die Evaluierung des umgesetzten Frameworks werden von allen vorgestellten Forschungsprojekten Metriken wie die Ausführungszeit betrachtet. Diese Arbeit greift die allgemeine Vorgehensweise der in \cite{rstensorflow2017} vorgestellten Experimente auf und wendet dies für andere Geräte an. Damit soll eine Validierung der in \cite{rstensorflow2017} vorgestellten Ergebnisse erzielt werden. Die nicht performante Nutzung des Arbeitsspeichers von RenderScript wird von den Autoren von \cite{rstensorflow2017} als ein möglicher Grund für die verschlechterte Ausführungszeit bei der Conv2D-Operation vermutet. Daher werden in diesem Paper weitere Metriken, wie beispielsweise die Speicherauslastung, betrachtet. 