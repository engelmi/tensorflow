\section{Optional: Optimierung der Conv2D-Operation}
\label{sec:optimierungconv2d}
+ unfolding ist mögliche optimierung
	- literatur: 
	- statt viele kleine, elementweise Operationen -> nur noch 1 große matrixmultiplikation
		- theoretisch diskutieren (größe der matrizen)
		- 1 große matmul, könnte die bereits bestehende matmul impl nutzen -> output matrix muss als tensor aufbereitet werden
	- wie könnte es umgesetzt werden?

\begin{itemize}
	\item{Kurze Beschreibung der Convolution-Operation (wie in \cite{stanford-CS231n})}
	\item{Ist-Analyse des Quellcodes der Conv2D-Operation}
	\item{Unfolding (\cite{conv2d-optimizing-unfolding}) als mögliche Optimierung von Conv2D für RSTensorFlow}
	%\begin{figure}[!t]
	%	\centering
	%	\includegraphics[width=3.0in]{simple-corr-model.png}
	%	\caption{A very simple model to describe the correlation between agile and safety. }
	%	\label{fig:simple-corr-model}
	%\end{figure}
	\item{Implementierung der (möglichen) Optimierung}
\end{itemize}